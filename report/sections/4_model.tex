\section{Model}
\label{sec:4_model}
% approx. 200-500 words
% \begin{itemize}
%     \item \textit{your network/ algorithm (do not spend too much text in explaining already existing models, focus on your solution),}
%     \item \textit{the pipeline if used any, including  tokenizer, featurizer, extractor, etc.}
%     \item \textit{Your baseline and the experiments you have tried}
% \end{itemize}
 
\subsection{Architecture}
\label{sec:4_arch}
We employed LSTMs\cite{hochreiter1997lstm} as backbone architecture, as they outperform Vanilla RNNs and GRUs in caputirng long-term dependencies, which is a key feature for language modelling. The model features the following components:
\begin{itemize}
    \item a \emph{word embedding} layer, which maps the input words to a sparse vector representation;
    \item a \emph{recurrent LSTM} layer, which processes the sequence of embeddings and outputs a sequence of hidden states;
    \item a \emph{fully-connected} layer, which maps the hidden states to a sequence of logits.
\end{itemize}
We have implemented two different architectures:
\begin{itemize}
    \item \emph{Baseline LSTM}: baseline model with a single-layer LSTM with \(300\) hidden units;
    \item \emph{Merity LSTM}: refined architecture featuring a set of regularization technique, which will be discussed later.
\end{itemize}

\subsection{Pipeline}
The complete experiment pipeline is structured as follows.
\begin{enumerate}
    \item \emph{Data loading}: the dataset is downloaded, extracted and loaded into memory.
    \item \emph{Data pre-processing}: in order to be ready to be fed to the model, the following operations are performed:
    \begin{itemize} 
        \item each sentence is tokenized by splitting on whitespaces;
        \item a \texttt{<eos>} token is appended to each sentence;
        \item the vocabulary mappings are created.
    \end{itemize}
    \item \emph{Data batching and  collation}: the dataset is padded to the length of the longest sequence in the batch before the batch is collated to be fed to the model. Additionally, depending on the experiment configuration the collate function can also perform the following operations:
    \begin{itemize}
        \item it can split sentences into chunks of a fixed length to implement TBPTT;
        \item it can partially shuffle sentences.
    \end{itemize}
    \item \emph{Model training}: the model is trained on the training set; we will also perform a validation epoch on the validation set in order to monitor the model learning behaviour and to tune the hyper-parameters.
    \item \emph{Model evaluation}: the model is evaluated on the test set according to the evaluation metrics and pipeline (see Sec.\ref{sec:5_eval}).
\end{enumerate}

\subsection{Regularization}
Sequence models are prone to overfitting, as they are trained on long sequences and they have to capture long-term dependencies. To cope with this problem, we employed the following regularization techniques suggested by \cite{merity2017regularizing}:
\begin{itemize}
    \item \emph{Variational Dropout}\cite{gal2016theoretically}: it samples a binary mask that drops the same units across time-steps, to ensure temporal consistency;
    \item \emph{Embedding Dropout}\cite{gal2016theoretically}: it drops some connections in the embedding layer; it is equivalent to dropping some words from the vocabulary, in order to prevent the model from relying too much on specific words;
    \item \emph{DropConnect}\cite{wan2013dropconnect}: it shuts down some connections in the hidden-to-hidden weight matrices by setting the weights of the target connections to $0$, to control overfitting on the recurrent connections;
    \item \emph{Weights Initialization}: initializing the weights can help the model to converge faster; specifically, we are initializing the LSTM layers according to the  \emph{Xavier Uniform} distribution\cite{glorot2010understanding} and the sparse embedding from a  \emph{Uniform} distribution in the range \([-0.1, 0.1]\);
    \item \emph{Weight Tying}\cite{press2017tying}: we tye the weights of the embedding layer and the output layer, in order to reduce model parameters;
    \item \emph{Gradient Clipping}: we clip the gradients to a maximum norm, in order to prevent gradient explosion;
    \item \emph{Partial Shuffle}\cite{press2019partially}: each sentence in a batch is rotated by a random offset, in order to improve generalization;
\end{itemize}\
\paragraph*{TBPTT}
Additionally, we used \emph{Truncated Backpropagation Through Time} (TBPTT) to train the model. As suggested in \cite{merity2017regularizing}, we are using $k_1 = k_2 = k$, i.e. we are splitting the sentences into chunks of a fixed length and backpropagating the error only through the chunks, allowing for a more efficient usage of training samples. Specifically, the split step $k$ is sampled from a Gaussian $\mathcal{N}(\textrm{seq}, \sigma^2)$ with probability $p=0.95$ and from $\mathcal{N}(\frac{\textrm{seq}}{2}, \sigma^2)$ with probability $1-p$. However, we are using $30$ in place of $70$ as value for $\textrm{seq}$. The motivation is to be found in the dataset statistics. Recalling Tab.\ref{tab:statistics}, we know that the mean sentence length is around $20$ and the length at the $75\%$ quantile is $27$, which means that the majority of the sentences are shorter than $30$ words and would not undergo any processing if $70$ was the mean of the Gaussian.

\paragraph*{Optimization}
We used mostly \emph{SGD} and its variation \emph{NT-ASGD}\cite{merity2017regularizing} as optimizers. We also performed some experiments with \emph{Adam}, but we did not observe any improvement in the model performance. We also employ a \emph{Learning Rate Scheduler} which reduces the learning rate by a factor of $0.5$ when the validation loss does not improve for $3$ epochs.

\subsection{Experiments}
Unless otherwise specified, we used a batch size of $128$ and we trained the models for $50$ epochs and used SGD with \emph{learning rate} $= 1$ and \emph{weight decay} $=1\cdot 10^{-6}$. All the experiments were conducted on a NVIDIA GeForce RTX 3050 GPU with 4GB of VRAM.
\paragraph*{Baseline} 
We perfomerd a first experiment with the default configuration to set a baseline for our study. Then, %before moving on to the refined architecture, 
we performed a second experiment were we simply add a \emph{Dropout} layer with $p=0.5$ after the LSTM layer.% We also tried to use \emph{Adam} as optimizer, but we did not observe any improvement in the model performance.
\paragraph*{Merity LSTM}
This is our implementation of the AWD-LSTM model proposed in \cite{merity2017regularizing}. We started by adding the dropout and regularization techniques individually and then combining them in different configurations, in order to understand their contribution and keep only the best performing ones, in a greedy fashion. Subsequently, motivated by the fact that NNs benefit from growing in depth in many tasks and that the aforementioned techniques proved to be able to present overfitting, we increased the size of the model by increasing the sizes of both the embedding and the hidden representation. Lastly, we picked the best performing model and we performed an extensive training experiment, where we let it train until the validation loss would stop decreasing.