
\section{Evaluation}
\label{sec:5_eval}
% approx. 400-800 words
% \begin{itemize}
%     \item \textit{The metrics you used}
%     \item \textit{Results on evaluation that you performed}
%     \item \textit{Comparison and differences between you model and the baseline}
%     \item \textit{Correct interpretation of errors and analysis}
% \end{itemize}
\subsection{Metrics}
In this work we are framing the learning problem as a multi-class classification problem, where the classes to predict are the words in the vocabulary. Hence, we are training the model to minimize the cross-entropy (CE) loss:
\begin{equation}
    CE(f(x; \theta), y) = -\sum_{i=1}^{N} y_i \log{f(x; \theta)_i}
\end{equation}
where $f(x; \theta)$ is the model, $x$ is the input sequence, $y$ is the target sequence, $\theta$ is the set of model parameters and $N$ is the number of classes in the vocabulary. 

The main metric we used to compare and evaluate the models is the \emph{perplexity} (PP). The reason is that PP is a well understood and established metric and it correlates well enough with the model performances on real-world tasks. Although the PP can be defined in multiple ways, it is convenient to use the definition of PP as the exponential of the cross-entropy (CE):
\begin{equation}
    \text{PP}(f(x; \theta), y) = \exp{CE(f(x; \theta), y)}
\end{equation}

% Intuitively, the PP can be interpreted as the number of classes that the model is considering to make a prediction . 
While the $CE$ gives a measure of how much the model is uncertain about the prediction, the PP can be interpreted as a measure of how many classes the model is considering to make a prediction (\emph{weighted branching factor}). The lower the PP, the less uncertain the model is on which word to predict, the better it is performing.

Additionally, we are further evaluating the best-performing model to excerpt deeper insights in its behaviour. Specifically, we are considering \emph{average PP per sequence length}, \emph{per word predicted vs. target counts difference} and \emph{precision, recall} and \emph{F1 score}. Lastly, we are giving a qualitative evaluation of the model by showing some examples of generated sequences.

\subsection{Results}
\begin{table}
    \begin{tabular}{lcc}
    \toprule
    Model& Validation & Test \\
    \midrule
    \texttt{baseline\_dropout} & 116.11 & 113.44 \\
    \texttt{merity\_ad} & 95.36 & 	91.29 \\
    \texttt{merity\_ad\_nohh} & 95.36 & 91.29 \\
    \texttt{merity\_ad\_nohh\_tbptt} & 90.38 &	87.46 \\
    \texttt{merity\_ad\_nohh\_1024} & 85.85 &	82.74 \\
    \texttt{merity\_ad\_nohh\_1024\_ps} & 87.40 &	83.18 \\
    \texttt{merity\_ad\_nohh\_1024\_ps\_long} & 83.01 &	79.86 \\
    \bottomrule
\end{tabular}
    \caption{Results (PP) of the experiments.}
    \label{tab:results}
\end{table}

\subsection{Analysis}